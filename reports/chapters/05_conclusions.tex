\documentclass[../main]{subfiles}

\begin{document}

\chapter{Conlusions}
In order to predict the average rating of a film, the required analysis pipeline was implemented:
\begin{center}
    Data Acquisition $\rightarrow$ Data Pre-Processing $+$ Visualization $\rightarrow$ Modeling $\rightarrow$ \\ $\rightarrow$ Performance Analysis $+$ Visualization
\end{center}
As a result of the performance analysis, it was seen that SVM and MovieNet (MLP) performed better than the other models considered, thus succeeding in predicting the class to which a film belongs.
Given the non-independence of the data characterising a film, it could be seen that GaussianNB failed to achieve good results.
It would therefore have been useful to have independence between the data belonging to the dataset so as to probably achieve better performance.
Finally, we would like to report an issue encountered during the implementation phase of the scikit-learn models, where the behaviour of the \textit{fit}, \textit{fit\_transform} and \textit{predict} methods using the imbalance-learn \textit{Pipeline} within the \textit{GridSearchCV} was unclear.

\end{document}